\chapter{Implementation}
Software used in all the experiments in this thesis is developed under the name PDT Word Sense Disambiguator. Entire system was implemented in Java programming language. Two third--party libraries were used: 
\begin{enumerate}
\item lucene\footnote{http://lucene.apache.org/java/docs/index.html}: for the purpose of text indexing,  released under the Apache Software License and
\item semantic vectors\footnote{http://code.google.com/p/semanticvectors/}: for the purpose of constructing random vectors, released under the BSD 2-Clause License   \footnote{http://www.opensource.org/licenses/bsd-license.php}
\end{enumerate}
Classes are separated into following packages: preprocessing, vectorModels, utils (contains data types, string manipulation classes, vector operation classes), evaluation and experiments. 

\section{Data flow}
In this section will be presented an outline of sequential steps taken by the system in order to train the
model, and then evaluate it on the test set. The options  mentioned here are dicussed in length in previous chapters,  therefore I will 
outline here only what is relevant for the data flow.There are 4 major stages that subsume a number of 
smaller, potential steps:
\begin{description}
\item \textbf{Obtaining and dividing the data} \hfill \\
	First step here is to extract the text from PDT1.0(which is in a form of fully annotated SGML text). 
Different meanings of ambiguous word are annotated differently, so this information is kept, while all
other words are extracted in their normal form. This part is vital because the model needs to be trained
on different meanings in order to be able to differentiate them during testing. 
	When the text is obtained it is divided into three sets: training, testing for development and final testing. Sentences from 
original data set are randomized before division. Training takes about 9/10 of the entire data set, while 1/10 is evenly 
distributed to remaining two sets. Evaluation is performed twice: during training, the model is trained 
on training set, and tested to development test set.  During testing, model is trained on train+testDev 
set and evaluated on final test set, which represents an unseen portion of data. Files that are trained on
and later tested on are run through the indexer. Every vector used in experiments is built straight from 
the indexer. 
\item \textbf{Preprocessing} \hfill \\
There are 4 preproccessing options (all optional): lowercasing, stemming, filtering words that belong to closed class,
and merging Czech lexical variants.  Lowercasing and word filtering take place in the indexer: lowercasing is just a matter of option, while for word filtering a stop word list needs to be passed. This list is to be found in the Apendix 1. Stemming and merging of czech variants are performed by the program directly on files. If one of these two options is used, it is applied before indexing.  
\item \textbf{Building the matrix}\hfill \\
After preprocessing indexer builds co-occurrence matrix (term--document). Documents passed to the 
indexer are determined by the system, and there are two options that can be set: number of sentences 
in one index document, and size of the context window found around ambiguous word. Any positive number can be set on either option, there is just a rule not to set number of sentences to a value 
larger than 1 when you want to build word--level documents. Choice of document size determines 
the "sensitivity" of the model, as is dicussed in Experiments chapter. During this phase all meanings of
ambiguous word are saved in the hash table. Elements of the hash table are items whose keys are 
ambiguous words while the values attached to the keys are lists which keep all the 
meanings of ambiguous word. This hash table is passed to evaluator during evaluation. 
\item \textbf{Normalizing the matrix: frequency weighting or dimensionality reduction }\hfill \\
Although in case of frequency weighting this step is performed during evaluation, it is more natural to
come before it, therefore I will describe it here. Frequency weighting is performed on elements of 
term--document matrix, and which weight scheme is applied depends on the model (either TF--IDF or 
PMI). Weighting is however not applied to RI model, due to its nature (see chapter on Random Indexing). RI model is constructed with the usage of semantic vectors package. After this step, the model is considered to be trained, and is ready for the evaluation.
\item  \textbf{Evaluation}\hfill \\
Evaluation is performed in several steps. First the entire test set is passed in order to extract all the 
ambiguous words and the contexts they are found in. For some reason in PDT1 some words are labeled
as ambiguous although they have only one meaning. These kinds of occurrences are inserted in 
previously mentioned hash table during matrix construction phase. If a an ambiguous word encountered 
in testing is determined to have only one meaning in training, it is not put into the test set. 
When test set instances are extracted, model compares distances between a test set instance of 
ambigous word and all other meanings to the context vector. Whichever meaning is the closest by 
cosine distance from the context vector is predicted to be the ''correct meaning". Term vectors are retrieved at this point, and in the case of TFIDF and PMI they are weighted on the spot. RI were
created during indexing and stored in $termvectors.bin$ file.   
\end{description}